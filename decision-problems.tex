\chapter{Decision Problems}

\section{Games Against Nature}

% this is a test
% this portion will come later in the section.
% we explain how to use Baye's formula to update a belief.
% need to change the problem context and add explanatory prose.
\emph{Updating a prior belief with new information.}
Your prior probability that a certain coin is biased to always land
heads up is 0.1. Now you toss the coin three times and observe that it
lands heads up every time. What is your posterior probability that the
coin is biased to always land heads up?  Use Baye's formula to compute
the posterior probability. Use the Binomial distribution to compute
the likelihood.

Let $B$ indicate that the coin is biased, and let $3H$ indicate an
outcome of three heads. We are given (or can determine)
\[
P(B) = 0.1, \quad P(3H \mid \overline{B}) = \left(\frac{1}{2}\right)^3, \quad P(3H \mid B) = 1
\]
We can use Baye's Theorem to compute the posterior probability.
\begin{align*}
P(B \mid 3H) &= \frac{P(B~\text{and}~3H)}{P(3H)} \\
&= \frac{P(3H \mid B)P(B)}{P(3H \mid B)P(B) + P(3H \mid \overline{B})P(\overline{B})} \\
&= \frac{(1)(0.1)}{(1)(0.1) + \left(\frac{1}{8}\right)(.9)} \\
&\approx 0.47
\end{align*}


\section{Games Against an Opponent}

% this will be the last portion of this section, where we bring together
% several concepts. Need to re-write this section: change the context
% of the problem and explain the methodology.
\emph{Nature as an adversary: a two-person zero-sum game.}
Merrill has a concession stand at Target Field for the sale of
sunglasses and umbrellas. This entrepreneur likes to make sales
regardless of the weather.  When it rains can sell about 500
umbrellas.  On a sunny day he can sell about 100 umbrellas and about
1000 sunglasses. Umbrellas cost him 50 cents and sell for \$1.
Sunglasses cost him 20 cents each and sell for 50 cents. Merrill is
willing to invest \$250 in the concession stand business.  All unsold
items represent a loss; there is no salvage value. 

Formulate Merrill's problem as a two-person zero--sum game. Merrill is
the row player and Nature is the column player. Merrill's strategy set
is \{buy inventory for rain, buy inventory for sun\}. Nature's strategy
set is \{rain, sun\}. The payoff entries represent the profit/loss.
Find an equilibrium strategy for Merrill. That is
to say, Merrill treats Nature as a strategic opponent and wants to
find an optimal inventory strategy that will yield a maximum expected
profit \emph{regardless} of the weather.

Would Merrill necessarily need to invest all \$250 into buying inventory
exclusively for rain or sun? In other words, does it seem possible that
Merrill could truly mix his two pure strategies and invest a portion
of the \$250 into each? The game is

\begingroup
\setlength{\tabcolsep}{9pt}
\renewcommand*{\arraystretch}{2}
\begin{tabularx}{4in}{YYYY}
& & \multicolumn{2}{c}{Nature} \\
& & Rain & Sun \\ \cline{3-4}
\multirow{2}{.5in}{Merrill} & \gtcol{Rain} & \gtcol{250} & \gtcol{-150} \\ \cline{3-4}
& \gtcol{Sun} & \gtcol{-150} & \gtcol{350} \\ \cline{3-4}
\end{tabularx}
\endgroup
\vspace{.1in}

The best strategy for Merrill is to mix buying for rain and buying
for sun in the ratio 5 to 4. These are the odds. To compute 
Merrill's expected profit (i.e. the value of the game) we use
Merrill's  equilibrium strategy against either of Nature's
pure strategies. Here is the payoff for Merrill against
Nature's strategy of Rain.

\[ \frac{5 \times (250) + 4 \times (-150)}{9} = \$72.22 \]

Merrill could play the odds and choose a pure strategy, but 
note that in this game it is possible for Merrill to physically
mix the strategies. He could invest 5/9 of his \$250 in
rainy--day inventory and invest 4/9 in sunny--day inventory.
So he buys
\[ \frac{5}{9} \left(500 \times .50\right) + \frac{4}{9} \left(100 \times .50\right) = \$161.11 \]
worth of umbrellas and
\[ \frac{4}{9} \left(1000 \times .20\right) = \$88.89 \]
worth of sunglasses so that he enjoys a steady profit of \$72.22.

\section{Exercises}
\begin{enumerate}
	
% This exercise is OK
\item \emph{Rules for decision--making under ignorance.}
  You have the opportunity to go on a
  blind date, but you are hesitant.  You are lonely and would like to
  find the love of your life; however, you dislike awkward
  situations. Furthermore, you find it difficult to estimate the
  probability that this particular blind date will turn out to be the
  love of your life, but you know this probability is
  non-negligible. To be a little more precise, you have the following
  values: finding the love of your life is worth 1000, being in an
  awkward date situation (i.e. being on a date and knowing that you
  will not see the person again) is worth -10, and staying home
  watching Netflix is worth zero.

\begin{enumerate}
\item Formulate a decision problem for deciding whether to go on the
blind date or to stay home.
\item Use the maximin rule to solve the problem.
\item Use the minimax regret rule to solve the problem.
\end{enumerate}

\begin{solution}
\bs The decision problem can be represented with the following table.
\\[.1in]
\begin{tabular}{ccc}
 \multicolumn{3}{c}{decision matrix} \\
 & find love & lots of awkward moments \\ \cline{2-3}
go on date & 1000 & -10 \\
decline date & 0 & 0 
\end{tabular}
\\[.1in] 
The maximin rule tells you to decline the date because it has
the best of all the worst possible outcomes. To use minimax regret, we
form the regret matrix.  \\[.1in]
\begin{tabular}{ccc}
 \multicolumn{3}{c}{regret matrix} \\
 & find love & lots of awkward moments \\ \cline{2-3}
go on date & 0 & -10 \\
decline date & -1000 & 0
\end{tabular}
\\[.1in] Minimax regret tells you to go on the date because the
possibility of not finding love has the most regret.
\end{solution}

% written by Emily
\item \emph{Gardening against nature.} A family is considering growing
  their own garden to save money on fresh vegetables. They have space
  in their yard for the garden but would need to purchase seeds and
  gardening supplies. The family is excited to grow a garden, but they
  know there are a lot of hungry rabbits in their neighborhood that
  might eat their plants before the family can harvest any vegetables
  from them. Money saved by the garden is shown in the following
  table.

\begin{tabular}{rcc}
& \multicolumn{2}{c}{State of Nature} \\
& $s_1$ & $s_2$ \\
& rabbits leave garden alone & rabbits eat garden \\ \cline{2-3}
plant garden & \$400 & -\$100\\
buy vegetables from store & 0 & 0
\end{tabular}

\begin{enumerate}
    \item If the probability that the rabbits leave the garden alone is 0.3, what decision is recommended for the family? What are the expected savings?
    
    \item The family has the option to purchase fast-growing plant seeds (the fast-growing seeds are the same price as regular seeds but they must buy the fast-growing seeds now if they want them because they are in high demand).  With these fast-growing seeds, the family can wait three more weeks to plant their garden. During that time, some scientists will finish their study on the appetites of the local rabbits, and the family will have a better idea about the probability that their garden is eaten by rabbits. They can return the seeds later for a partial refund if they do not use them.
    Let $L$ represent the event the rabbits have large appetites and let $S$ represent
    the event that rabbits have small appetites. Then
    \[
    \begin{matrix}
    P(L)=0.60, & P(s_1 \mid L)=0.15, & P(s_2 \mid L)=0.85,\\
    P(S)=0.40, & P(s_1 \mid S)=0.79, & P(s_2 \mid S)=0.21.
    \end{matrix}
    \]
    What is the optimal decision strategy if the family purchases the fast-growing seeds so they can wait and learn more about the rabbit appetites before making a decision?
    
    \item If \$40\ of the fast-growing seed purchase is non-refundable, should the family purchase the fast-growing seeds? Why or why not? What is the maximum non-refundable amount the family should pay to get the fast-growing seeds?
\end{enumerate}

\begin{solution}
\bs For part a), the expected savings when planting the garden are
\[ 400 \times 0.3  - 100 \times 0.7 = \$50. \]
The savings from not planting the garden are \$0, so based on expected value,
the best decision is to plant the garden.

For part b), if the rabbits have large appetites ($L$), then planting 
the garden would result in -\$25 of expected savings.
If the rabbits have small appetites ($S$), then planting the garden will result in \$295 
of expected savings. 

If L, \[ \$100 \times 0.15 - \$100 \times 0.85 = -\$25 \]
If S, \[ \$400 \times 0.79 - \$100 \times 0.21 = \$295 \]

Not planting will always result in \$0 of savings. 
The optimal decision strategy is to plant the garden if $S$ and buy vegetables 
from the store if $L$.

For part c), we use the optimal decision for each possible event $L$ and $S$. 
The expected savings from purchasing the fast-growing seeds
(but before actually purchasing the seeds) are
\[ \$0 \times 0.60 + \$295 \times 0.40 = \$118 \]
The maximum non-refundable amount that the family should be willing 
to pay for the fast-growing seeds is
\[ \$118 - \$50 = \$68 \]
\end{solution}

% this problem is OK
\item \emph{Using Baye's formula to update a prior belief.}
Curling is a sport in which players slide a stone over ice toward a
target. The association governing the sport has implemented drug
testing. It is believed that 15\% of all curlers use banned drugs to
enhance performance. If a player uses banned drugs, the association
may take away any prizes that the player has won; however, it
undesirable to falsely accuse someone of using banned substances.  The
utilities for each decision and state of nature are

\begin{center}
\begin{tabular}{rrrr}
& drug use & no drug use \\ \cline{2-3}
take away prizes & -100 & -1000 \\
do not & -600 & 0 
\end{tabular}
\end{center}

Notice that there is a small dis-utility for taking prizes away from a drug user
due to bad publicity for the sport. The test to detect drug use is
less than 100\% reliable. In particular, if $D$ indicates that a 
player uses banned drugs, and $+$/$-$ indicate a positive/negative
test result, then the true positive rate and the true negative rate
are
\[
P(+ \mid D) = .97 \quad \text{and} \quad P(- \mid \overline{D}) = .97,
\]
respectively, Given the utilities and the accuracy of the test, what is the best
decision if a player has a positive test result? (The association
wants to maximize expected utility.)

\begin{solution}
\bs First we update the probability of drug use via Baye's formula.
\begin{align*}
P(D \mid +) &= \frac{P(D \cap +)}{P(+)} \\
&= \frac{P(+ \mid D)P(D)}{P(+ \mid D)P(D) + P(+ \mid \overline{D})P(\overline{D})} \\
&= \frac{.97 \times .15}{.97 \times .15 + .03 \times .85} \\
&= .851
\end{align*}
and then we can compute $P(\overline{D} \mid +) = 1 - P(D \mid +) = .149$.
Using these posterior probabilities, the expected utilities are
\begin{align*}
E(\text{take away}) &= (-100)(.851) + (-1000)(.149) = -234 \\
E(\text{do not}) &= (-600)(.851) = -511
\end{align*}
The best decision is to take away prizes when a player tests positive.
\end{solution}

% this problem needs to be re-written. the main idea is that decisions
% are made based on expected value and that we can perform a
% sensitivity analysis and obtain a threshold on a decision, i.e. make
% decision x if the probability of the state of nature is greater than
% p.
\item \emph{Decisions under risk and sensitivity analysis.}
  Modern forest
  management uses controlled fires to reduce fire hazards and to
  stimulate new forest growth. Management has the option to postpone
  or plan a burning. In a specific forest tract, if burning is
  postponed, a general administrative cost of \$300 is incurred. If a
  controlled burning is planned, there is a 50\% chance that good
  weather will prevail and burning will cost \$3200. The results of
  the burning may be either successful with probability .6 or marginal
  with probability .4. Successful execution will result in an
  estimated benefit of \$6000, and marginal execution will provide
  only \$3000 in benefits. If the weather is poor, burning will be
  canceled incurring a cost of \$1200 and no benefit.

  \begin{enumerate}
  \item On the basis of expected value, should a burning
    be planned or postponed?
  \item How does the decision change with the probability of good/poor
    weather? \label{sen} In other words, you should perform a
    sensitivity analysis.
  \end{enumerate}

\begin{solution}
\bs
A decision table for this problem is

\begin{center}
\begin{tabular}{rrr}
    & $\frac{1}{2}$ & $\frac{1}{2}$ \\
    & good weather & poor weather \\ \hline
    postpone burn & -\$300 & -\$300 \\
    plan burn & \$1600 & -\$1200
\end{tabular}
\end{center}

Note that the payoff of \$1600 for a planned burn and resulting good weather is
\[
-\$3200 + (.6)(\$6000) + (.4)(\$3000) = \$1600
\]

The expected payoff of each decision is
\begin{align*}
&E(\text{postpone}) = -\$300 \\
&E(\text{plan}) = \left(\frac{1}{2}\right)(\$1600)
   + \left(\frac{1}{2}\right)(-\$1200) = \$200
\end{align*}
Management should plan the burn. For part~\ref{sen}, it
makes sense that as the probability of poor weather
increases, the expected value of planning a burn will
decrease. Specifically, let $p$ represent the probability
of poor weather.
\begin{align*}
  E(\text{plan}) &= (1-p)1600 + p(-1200) \\
  &= 1600 - 2800p
\end{align*}
$E(\text{plan})$ is decreasing in $p$. Management should
plan a burn as long as
\begin{align*}
  E(\text{plan}) &\geq E(\text{postpone}) \\
  1600 - 2800p &\geq -300 \\
  -2800p &\geq -1900 \\
  p &\leq \frac{19}{28} \approx .68
\end{align*}
Unless the probability of poor weather is greater than .68,
planning to burn is the best decision. 
\end{solution}

=======
% Emily - expected value and sensitivity analysis
\item \emph {Decisions under risk and sensitivity analysis.} 
	The employees at a popular outdoor furniture company predict that their 
	sales will double this coming year. The company is already producing the 
	maximum amount of furniture possible in their current facility. They are 
	considering building another manufacturing facility to be able to meet the 
	predicted increase in demand. The facility will cost \$500,000 to build. If 
	the demand doubles as predicted, their income will increase by \$800,000. 
	If the demand only slightly increases, their income will increase by \$250,000. 
	If the facility is not built, the company will lose \$50,000 because of the 
	time workers will have to spend talking to upset customers to explain why 
	furniture pieces are out-of-stock. The change in demand will be determined by 
	the weather that year because more people want to buy outdoor furniture if they 
	can enjoy the good weather outside. There is a 0.55 chance of good weather, 
	which will result in a double in demand. There is a 0.45 chance of poor 
	weather, which will result in only a slight increase in demand.

\begin{enumerate}
	\item Should the facility by built? Use expected value to make your decision.
	
	\item How does the decision change with the probability of good/poor
	weather? Perform a sensitivity analysis to answer.
	
\end{enumerate}

\begin{solution}
	\bs The decision table for this problem is
	\begin{center}
		\begin{tabular}{rrr}
			& $\frac{55}{100}$ & $\frac{45}{100}$ \\
			& good weather & poor weather \\ \hline
			build facility & \$300,000 & -\$250,000 \\
			do not build & -\$50,000 & -\$50,000
		\end{tabular}
	\end{center}

The expected payoff of each decision is
\begin{align*}
&E(\text{do not build}) = -\$50,000 \\
&E(\text{build}) = \left(\frac{55}{100}\right)(\$300,000)
+ \left(\frac{45}{100}\right)(-\$250,000) = \$52,500
\end{align*}
The company should build the facility. As the probability of poor weather increases, the expected value of building the new facility decreases. 
Let $p$ represent the probability of poor weather.
\begin{align*}
E(\text{build}) &= (1-p)300,000 + p(-250,000) \\
&= 300,000 - 550,000p
\end{align*}
The company should build that facility as long as
\begin{align*}
E(\text{build}) &\geq E(\text{don't build}) \\
300,000 - 550,000p &\geq -50,000 \\
-550,0000p &\geq -350,000 \\
p &\leq \frac{7}{11} \approx .64
\end{align*}
Building the facility is the best decision unless the probability of poor weather is greater than .64.	

\end{solution}

% written by Emily
\item \emph{Elimination of dominated strategies.}
Two street vendors, A and B, are located near a major tourist attraction. 
The proportion of customers
captured by each vendor depends on the merchandise sold by that vendor and by
her competitor. A customer gained by one is lost to the other. Each vendor
can stock one of the following: clothing, ice cream, or souvenirs.
The possible strategies and proportion of customers captured are as follows.

\begin{tabular}{l}
If both shops sell souvenirs, A captures 75\% of the customers.\\
If both shops sell clothing, A and B split the customers evenly.\\
If both shops sell ice cream, A and B split the customers evenly.\\
If B sells ice cream and A sells souvenirs, A captures 10\%.\\
If B sells clothing and A sells ice cream, A captures 90\%.\\
If B sells souvenirs and A sells clothing, A captures 10\%.\\
If A sells clothing and B sells ice cream, A captures 100\%.\\
If A sells souvenirs and B sells clothing, A captures 75\%.\\
If A sells ice cream and B sells souvenirs, A captures 40\%.
\end{tabular}

\setlength{\parindent}{0cm}
Model the decision of each vendor as two-person zero-sum game
and find a solution by elimination of dominated strategies.

\begin{solution}
\bs The game is

\begingroup
\setlength{\tabcolsep}{9pt}
\renewcommand*{\arraystretch}{2}
\begin{tabularx}{4.5in}{YYYYY}
& & \multicolumn{3}{c}{B} \\
& & clothing & ice cream & souvenirs \\ \cline{3-5}
\multirow{3}{.25in}{A} & \gtcol{clothing} & \gtcol{.50} & \gtcol{1} & \gtcol{.10} \\ \cline{3-5}
& \gtcol{ice cream} & \gtcol{.90} & \gtcol{.50} & \gtcol{.40} \\ \cline{3-5}
& \gtcol{souvenirs} & \gtcol{.75} & \gtcol{.10} & \gtcol{.25} \\ \cline{3-5}
\end{tabularx}
\endgroup
\vspace{.1in}

For A, ice cream strictly dominates souvenirs and for B, souvenirs strictly dominates
clothing, leaving a $2 \times 2$ game.

\begingroup
\setlength{\tabcolsep}{9pt}
\renewcommand*{\arraystretch}{2}
\begin{tabularx}{4in}{YYYY}
& & \multicolumn{2}{c}{B} \\
& & ice cream & souvenirs \\ \cline{3-4}
\multirow{2}{.25in}{A} & \gtcol{clothing} & \gtcol{1} & \gtcol{.10} \\ \cline{3-4}
& \gtcol{ice cream} & \gtcol{.50} & \gtcol{.40} \\ \cline{3-4}
\end{tabularx}
\endgroup
\vspace{.1in}

Now for B, souvenirs dominates ice cream. Then, A will choose to sell ice cream
over clothing. So, the
best strategies for A and B are to sell ice cream and souvenirs, respectively. 
Using this pair of strategies, A will capture 40\% of the customers.
\end{solution}

% this problem definitely needs to be re-written.  it is unfair to
% students for whom english is a 2nd language.  the main idea is that
% we have a 2x3 game, but for the column player, one strategy can be
% eliminated by domination and so it's really a 2x2 game and we can
% solve it by hand. it's OK to have one or two exercises that are
% a little quirky.
\item \emph{A two-person zero-sum game.}
Two friends, Goldsen and Kershaw get into a battle of
  words. Kershaw, hoping to make a little money on the argument,
  suggests that they create words, or try to, according to the scheme
  he outlines. Kershaw says the following to Goldsen:

\begin{quote}
Suppose you choose either the letter \emph{a} or the letter
  \emph{i}, and, independently and at the same time, I will choose
  \emph{f}, \emph{t}, or \emph{x}. If the two letters chosen form a
  word, I will pay you \$1, plus a \$3 bonus if the word is a noun or
  pronoun. In the event that the letters chosen don't form a word, you
  pay me \$2.
\end{quote}

Formulate Kershaw's scheme as a 2--by--3 game and determine the best
strategy for each player. Is this a fair game?

\begin{solution}
\bs The game is

\begingroup
\setlength{\tabcolsep}{9pt}
\renewcommand*{\arraystretch}{2}
\begin{tabularx}{4in}{YYYYY}
& & \multicolumn{3}{c}{Kershaw} \\
& & \emph{f} & \emph{t} & \emph{x} \\ \cline{3-5}
\multirow{2}{.5in}{Goldsen} & \gtcol{\emph{a}} & \gtcol{-2} & \gtcol{1} & \gtcol{4} \\ \cline{3-5}
& \gtcol{\emph{i}} & \gtcol{1} & \gtcol{4} & \gtcol{-2} \\ \cline{3-5}
\end{tabularx}
\endgroup
\vspace{.1in}

Note that there is no saddle point. Also note that Kershaw's strategy
of \emph{t} is dominated by \emph{f}, so there is no reason that 
he would choose \emph{t}. The reduced game is

\begingroup
\setlength{\tabcolsep}{9pt}
\renewcommand*{\arraystretch}{2}
\begin{tabularx}{4in}{YYYY}
& & \multicolumn{2}{c}{Kershaw} \\
& & \emph{f} & \emph{x} \\ \cline{3-4}
\multirow{2}{.5in}{Goldsen} & \gtcol{\emph{a}} & \gtcol{-2} & \gtcol{4} \\ \cline{3-4}
& \gtcol{\emph{i}} & \gtcol{1} & \gtcol{-2} \\ \cline{3-4}
\end{tabularx}
\vspace{.1in}
\endgroup

Using our scheme outlined in class, Goldsen should play a mixture of
\emph{a} and \emph{i} according to the odds 3:6 (or equivalenty
1:2). Kershaw should mix \emph{f} and \emph{x} according to the odds
6:3 (or equivalently 2:1). The value of the game is
\[
\frac{1 \times (-2) + 2 \times 1}{3} = 0,
\]
and so this is a fair game. If Kershaw uses \emph{t}, Goldsen will 
get, on average
\[
\frac{1 \times 1 + 2 \times 4}{3} = \$3.
\]
\end{solution}

% written by Emily to replace Goldsen and Kershaw battle of words
\item \emph{A two-person zero-sum game.}
   A professional football player believes that the team he plays for should be allocating more money to the salaries of the players, so he wants his contract
   to be changed to pay him more. His two options are to play in the upcoming season or not play in the upcoming season and hope the team will negotiate with him. The team knows that he is a valuable player but does not want to pay him more or go through the process of negotiations. The team has come up with three options to deal with the situation: negotiate, refuse to negotiate and play the season without him, or increase the player's salary by a set amount with no other negotiations. Keep in mind that the player wants to maximize his salary, and the team wants to minimize their costs, which means keep salaries as low as possible. The payouts are as follows (in millions of dollars):
  
  
  \begin{tabular}{l}
  	If the player plays and the team does not negotiate, the player's \\
  	salary will not change. \\
  	If the player does not play and the team does not negotiate, the \\
  	player will find a different job as a broadcaster for a pay\\
  	increase of 1 because he is such a well-known person. This is \\
  	bad publicity for the team and hurts their jersey sales. \\
   	If the player plays but the team still negotiates, the \\
   	player will end up with a pay increase of 3. \\
   	If the player had chosen to not play and the team negotiates, \\
   	the negotiations will go poorly and the player will end up \\
   	kicked off the team and having to deal with costs related to \\
   	poor publicity. \\
    If the team decides to set an increase with no negotiations, \\
    the player will end up with a positive 2 no matter what he \\
    chooses to do.	
  \end{tabular}

  
  Formulate the 2-by-3 decision matrix in terms of changes to the player's salary. 
  Determine the best strategy for the player and for the team. 
  Who is most likely to come out ahead in this situation?
  
  \begin{solution}
  	\bs The game is
  	
  	\begingroup
  	\setlength{\tabcolsep}{9pt}
  	\renewcommand*{\arraystretch}{2}
  	\begin{tabularx}{4in}{YYYYY}
  		& & \multicolumn{3}{c}{Team} \\
  		& & \emph{negotiate} & \emph{don't negotiate} & \emph{double salary} \\ \cline{3-5}
  		\multirow{2}{.5in}{Player} & \gtcol{play} & \gtcol{3} & \gtcol{0} & \gtcol{2} \\ \cline{3-5}
  		& \gtcol{\emph{don't play}} & \gtcol{-2} & \gtcol{1} & \gtcol{2} \\ \cline{3-5}
  	\end{tabularx}
  	\endgroup
  	\vspace{.1in}
  	
  	There is no saddle point in this game. The team's strategy
  	of a set increase is dominated by the strategy to not negotiate, so there is no financial reason that 
  	they would chose to offer a pre-determined increase without negotiations. The reduced game is
  	
  	\begingroup
  	\setlength{\tabcolsep}{9pt}
  	\renewcommand*{\arraystretch}{2}
  	\begin{tabularx}{4in}{YYYY}
  		& & \multicolumn{2}{c}{Team} \\
  		& & {negotiate} & {don't negotiate} \\ \cline{3-4}
  		\multirow{2}{.5in}{Player} & \gtcol{play} & \gtcol{3} & \gtcol{0} \\ \cline{3-4}
  		& \gtcol{don't play} & \gtcol{-2} & \gtcol{1} \\ \cline{3-4}
  	\end{tabularx}
  	\vspace{.1in}
  	\endgroup
  	
  	Using the scheme outlined in class, the player should play or not play according to the odds 1:1. The team should negotiate or not negotiate according to the odds 1:5. The value of the game is 
  	\[
  	\frac{1 \times (3) + 1 \times (-2)}{2} = 1/2,
  	\]
  	so the player is more likely to come out ahead. 

  
  \end{solution}
  

% I like this problem but it was mostly taken from another book, so
% needs to be re-written. I added the last part of the question
% ... the idea that as long as one player sticks to the optimal
% strategy, then it makes no difference what the other player does. i
% would like to keep that idea.
\item \emph{Cops and Robbers.} A police officer must decide whether
  to patrol the streets or hang out at the coffee shop.
  Simultaneously, a robber must decide whether to prowl the streets or
  stay in his hideaway. If the police officer patrols and the robber
  prowls, then the officer will surely make an arrest with a payoff of
  10 for the officer's reputation. If the officer is in the coffee
  shop while the robber prowls, the payoff (in the form of damage to
  the officer's reputation) is -1. If the robber stays hidden, then
  the payoff to the officer is zero to patrol and 5 to hang out
  in the coffee shop.
\begin{enumerate}
\item  Formulate this decision problem as a two--person
  zero--sum game and determine the optimal strategies for the
  officer and robber.
\item Compute the value of the game. \label{val}
\item Suppose that the robber, who did not take a course in game
  theory, has a police radio. He overhears the officer, who did well
  in her course on game theory, telling her partner of her mixed
  strategy. That is to say, the robber knows the probabilities with
  which the officer will patrol or get coffee (but he does not know
  which pure strategy the officer will actually choose). The robber
  thinks ``Hmmm. That officer is more likely to get coffee than to
  patrol. Since I know this, I will prowl the streets and my chances
  of not getting caught are improved.'' Is the robber's argument
  valid? In other words, does the value of the game change if the
  robber knows the officer's mixing probabilities (and the officer
  sticks to them)? \label{knowledge}
\end{enumerate}

\begin{solution}
\bs
The game is

\begingroup
\setlength{\tabcolsep}{9pt}
\renewcommand*{\arraystretch}{2}
\begin{tabularx}{3.25in}{YYYY}
& & \multicolumn{2}{c}{Robber} \\
& & prowl & hide \\ \cline{3-4}
\multirow{2}{.5in}{Officer} & \gtcol{patrol} & \gtcol{10} & \gtcol{0} \\ \cline{3-4}
& \gtcol{coffee} & \gtcol{-1} & \gtcol{5} \\ \cline{3-4}
\end{tabularx}
\endgroup
\vspace{.1in}

Note that there is no saddle point, so the best strategy
is a mixed one. According to our scheme given in class, the
officer's best strategy is to randomize between patrolling and getting
coffee in the ratio 6 to 10, while the robber should mix his
strategy of prowling/hiding in the ratio 5 to 11. The corresponding
probabilities are $(3/8,\,5/8)$ for the officer and $(5/16,\,11/16)$
for the robber.

Part \ref{val}. To compute the value of the game, note that
when the officer patrols, she receives a payoff of 10 with
probability 5/16 and payoff of zero with probability 11/16.
The value of the game is
\[ \frac{5 \times 10 + 11 \times 0}{16} = \frac{50}{16} = 3.125 \]
On average the officer comes out ahead.

Part \ref{knowledge}, the robber's thought process is not valid.
The value of the game does not change as long as one player
sticks to the optimal mixed strategy.
\end{solution}

>>>>>>> Stashed changes
% written by Emily to replace cops and robbers
\item \emph{Marketing strategies.} Two peanut butter companies,
  Doodle's and Lola's, are deciding on their marketing strategy for
  the upcoming year. They know that they are each other's main
  competitor and that the demand for peanut butter is relatively
  constant, so a gain in sales for Doodle's is a loss of sales for
  Lola's. Each company has their standard packaging for peanut butter
  and a new innovative packaging for peanut butter. Both companies
  may produce and sell both types of packaging.  If both
  companies choose to market only their innovative packaging, Doodle's
  will gain an extra 2\%\ of the market's sales.  If both companies
  choose to market their standard packaging, Doodle's will loose 2\%\
  of the market.  If Doodle's markets their innovative product and
  Lola's markets their standard product, Doodle's will gain 10\%\
  of the market.  If Lola's markets their innovative product and
  Doodle's markets their standard product, Doodle's will gain 8\%\
  of the market.
 
\begin{enumerate}
\item Formulate this decision problem as a two-person zero-sum game
  and determine the optimal marketing strategy for each company. Note
  that they are able to change their advertising throughout the year,
  so a mixed strategy is possible.
\item Compute the value of the game. \label{val}
\item Suppose that a member of the Doodle's marketing team quits their
  job and goes to work for the Lola's. She tells her
  new co-workers about the strategy that Doodle's is planning to
  use. She is able to tell them the probabilities with which Doodle's
  will market their standard product and their innovative product. The
  Lola's marketing team now knows that Doodle's is more likely to
  market the innovative product than the standard product. Armed with this
  knowledge, they choose to only market their innovative product. Is
  Lola's argument valid? In other words, does the value of the game
  change if the Lola's marketing team knows the Doodle's marketing
  team's mixing probabilities?
\label{knowledge}
\end{enumerate}

\begin{solution}
	\bs
	The game is

\begingroup
\setlength{\tabcolsep}{9pt}
\renewcommand*{\arraystretch}{2}
\begin{tabularx}{3.25in}{YYYY}
& & \multicolumn{2}{c}{Lola's} \\
& & standard & innovative \\ \cline{3-4}
\multirow{2}{.5in}{Doodle's} & \gtcol{standard} & \gtcol{-2} & \gtcol{8} \\ \cline{3-4}
& \gtcol{innovative} & \gtcol{10} & \gtcol{2} \\ \cline{3-4}
\end{tabularx}
\endgroup
\vspace{.1in}

Note that there is no saddle point, so the best strategy is mixed.
Doodle's should mix the strategies standard and innovative
in the ratio 4 to 5, while Lola's should mix their strategies of
standard to innovative in the ratio 1 to 2.  The corresponding
probabilities are $(4/9,\,5/9)$ for Doodle's and $(1/3,\,2/3)$ for
Lola's.

To compute the value of the game, note that when the
Doodle's markets the standard product, they receive a payoff of -2
with probability 1/3 and payoff of 8 with probability 2/3.  The value
of the game is
\[ \frac{1 \times -2 + 2 \times 8}{3} = \frac{14}{3} = 4.67 \]
On average Doodle's comes out ahead.

Regarding part \ref{knowledge}), the team's thought process is not
valid.  As long as one player sticks to the optimal mixed strategy,
the value of the game does not change.
\end{solution}

% written by Emily to replace the silver dollar
\item \emph{The birthday gift.} Liam's birthday is coming up and he
  can't wait to see what he will get as a gift. Liam's parents want
  the gift to be a surprise, but they always hide gifts in either the
  kitchen or the basement. Liam plans to search for the gift when his
  parents are busy, but he knows that even if he searches the room
  that contains the gift, he may not find it. If the gift is hidden in
  the kitchen and Liam searches the kitchen, he will find the gift
  with probability 0.75.  If the gift in hidden in the basement and he
  searches the basement, then he will find it with probability 0.5.
  If he searches the wrong room, there is no way he will find the
  gift. Assume that the payoff to Liam for finding the gift early is
  the same as the payoff to the parents of keeping the gift a
  surprise.  Formulate this game as a two-person, zero-sum game. Liam
  is the row player and his parents are the column player. Find the
  optimal strategies for both players. \label{sda}

\begin{solution}
  \bs Liam has two possible actions for this game: search the kitchen
  or search the basement. His parents also have two options: hide the
  gift in the kitchen or hide the gift in the basement.  The game is

\begingroup
\setlength{\tabcolsep}{9pt}
\renewcommand*{\arraystretch}{2}
\begin{tabularx}{4in}{YYYY}
& & \multicolumn{2}{c}{Parents} \\
& & hide in kitchen & hide in basement \\ \cline{3-4}
\multirow{2}{.5in}{Liam} & \gtcol{search kitchen} & \gtcol{3/4} & \gtcol{0} \\ \cline{3-4}
& \gtcol{search basement} & \gtcol{0} & \gtcol{1/2} \\ \cline{3-4}
\end{tabularx}
\endgroup
\vspace{.1in}

and the optimal strategy is the same for both players. Each
should play a mixture of $(2/5,~3/5)$.
\end{solution}

\end{enumerate}
