\chapter{Decision Problems}

\section{Games Against Nature}

% this portion will come later in the section.
% we explain how to use Baye's formula to update a belief.
% need to change the problem context and add explanatory prose.
\emph{Updating a prior belief with new information.}
Your prior probability that a certain coin is biased to always land
heads up is 0.1. Now you toss the coin three times and observe that it
lands heads up every time. What is your posterior probability that the
coin is biased to always land heads up?  Use Baye's formula to compute
the posterior probability. Use the Binomial distribution to compute
the likelihood.

Let $B$ indicate that the coin is biased, and let $3H$ indicate an
outcome of three heads. We are given (or can determine)
\[
P(B) = 0.1, \quad P(3H \mid \overline{B}) = \left(\frac{1}{2}\right)^3, \quad P(3H \mid B) = 1
\]
We can use Baye's Theorem to compute the posterior probability.
\begin{align*}
P(B \mid 3H) &= \frac{P(B~\text{and}~3H)}{P(3H)} \\
&= \frac{P(3H \mid B)P(B)}{P(3H \mid B)P(B) + P(3H \mid \overline{B})P(\overline{B})} \\
&= \frac{(1)(0.1)}{(1)(0.1) + \left(\frac{1}{8}\right)(.9)} \\
&\approx 0.47
\end{align*}


\section{Games Against an Opponent}

% this will be the last portion of this section, where we bring together
% several concepts. Need to re-write this section: change the context
% of the problem and explain the methodology.
\emph{Nature as an adversary: a two-person zero-sum game.}
Merrill has a concession stand at Target Field for the sale of
sunglasses and umbrellas. This entrepreneur likes to make sales
regardless of the weather.  When it rains can sell about 500
umbrellas.  On a sunny day he can sell about 100 umbrellas and about
1000 sunglasses. Umbrellas cost him 50 cents and sell for \$1.
Sunglasses cost him 20 cents each and sell for 50 cents. Merrill is
willing to invest \$250 in the concession stand business.  All unsold
items represent a loss; there is no salvage value. 

Formulate Merrill's problem as a two-person zero--sum game. Merrill is
the row player and Nature is the column player. Merrill's strategy set
is \{buy inventory for rain, buy inventory for sun\}. Nature's strategy
set is \{rain, sun\}. The payoff entries represent the profit/loss.
Find an equilibrium strategy for Merrill. That is
to say, Merrill treats Nature as a strategic opponent and wants to
find an optimal inventory strategy that will yield a maximum expected
profit \emph{regardless} of the weather.

Would Merrill necessarily need to invest all \$250 into buying inventory
exclusively for rain or sun? In other words, does it seem possible that
Merrill could truly mix his two pure strategies and invest a portion
of the \$250 into each? The game is

\begingroup
\setlength{\tabcolsep}{9pt}
\renewcommand*{\arraystretch}{2}
\begin{tabularx}{4in}{YYYY}
& & \multicolumn{2}{c}{Nature} \\
& & Rain & Sun \\ \cline{3-4}
\multirow{2}{.5in}{Merrill} & \gtcol{Rain} & \gtcol{250} & \gtcol{-150} \\ \cline{3-4}
& \gtcol{Sun} & \gtcol{-150} & \gtcol{350} \\ \cline{3-4}
\end{tabularx}
\endgroup

\vspace{.2in}
The best strategy for Merrill is to mix buying for rain and buying
for sun in the ratio 5 to 4. These are the odds. To compute 
Merrill's expected profit (i.e. the value of the game) we use
Merrill's  equilibrium strategy against either of Nature's
pure strategies. Here is the payoff for Merrill against
Nature's strategy of Rain.

\[ \frac{5 \times (250) + 4 \times (-150)}{9} = \$72.22 \]

Merrill could play the odds and choose a pure strategy, but 
note that in this game it is possible for Merrill to physically
mix the strategies. He could invest 5/9 of his \$250 in
rainy--day inventory and invest 4/9 in sunny--day inventory.
So he buys
\[ \frac{5}{9} \left(500 \times .50\right) + \frac{4}{9} \left(100 \times .50\right) = \$161.11 \]
worth of umbrellas and
\[ \frac{4}{9} \left(1000 \times .20\right) = \$88.89 \]
worth of sunglasses so that he enjoys a steady profit of \$72.22.

\section{Exercises}
\begin{enumerate}

% This exercise is OK
\item Rules for decision--making under ignorance (that is to
  say, decision--making without probabilities). You have the opportunity to go on a
  blind date, but you are hesitant.  You are lonely and would like to
  find the love of your life; however, you dislike awkward
  situations. Furthermore, you find it difficult to estimate the
  probability that this particular blind date will turn out to be the
  love of your life, but you know this probability is
  non-negligible. To be a little more precise, you have the following
  values: finding the love of your life is worth 1000, being in an
  awkward date situation (i.e. being on a date and knowing that you
  will not see the person again) is worth -10, and staying home
  watching Netflix is worth zero.

\begin{enumerate}
\item Formulate a decision problem for deciding whether to go on the
blind date or to stay home.
\item Use the maximin rule to solve the problem.
\item Use the minimax regret rule to solve the problem.
\end{enumerate}

\begin{solution}
\bs The decision problem can be represented with the following table.
\\[.2in]
\begin{tabular}{ccc}
 \multicolumn{3}{c}{decision matrix} \\
 & find love & lots of awkward moments \\ \cline{2-3}
go on date & 1000 & -10 \\
decline date & 0 & 0 
\end{tabular}
\\[.2in] 
The maximin rule tells you to decline the date because it has
the best of all the worst possible outcomes. To use minimax regret, we
form the regret matrix.  \\[.2in]
\begin{tabular}{ccc}
 \multicolumn{3}{c}{regret matrix} \\
 & find love & lots of awkward moments \\ \cline{2-3}
go on date & 0 & -10 \\
decline date & -1000 & 0
\end{tabular}
\\[.2in] Minimax regret tells you to go on the date because the
possibility of not finding love has the most regret.
\end{solution}

% written by Emily
\item \emph{Gardening against nature.} A family is considering growing their own garden to save money on fresh vegetables. They have space in their yard for the garden but would need to purchase seeds and gardening supplies. The family is excited to grow a garden, but they know there are a lot of hungry rabbits in their neighborhood that might eat their plants before the family can harvest any vegetables from them. Money saved by the garden is shown in the following table.

\begin{tabular}{rcc}
& \multicolumn{2}{c}{State of Nature} \\
& $s_1$ & $s_2$ \\
& rabbits leave garden alone & rabbits eat garden \\ \cline{2-3}
plant garden & \$400 & -\$100\\
buy vegetables from store & 0 & 0
\end{tabular}

\begin{enumerate}
    \item If the probability that the rabbits leave the garden alone is 0.3, what decision is recommended for the family? What are the expected savings?
    
    \item The family has the option to purchase fast-growing plant seeds (the fast-growing seeds are the same price as regular seeds but they must buy the fast-growing seeds now if they want them because they are in high demand).  With these fast-growing seeds, the family can wait three more weeks to plant their garden. During that time, some scientists will finish their study on the appetites of the local rabbits, and the family will have a better idea about the probability that their garden is eaten by rabbits. They can return the seeds later for a partial refund if they do not use them.
    Let $L$ represent the event the rabbits have large appetites and let $S$ represent
    the event that rabbits have small appetites. Then
    \[
    \begin{matrix}
    P(L)=0.60, & P(s_1 \mid L)=0.15, & P(s_2 \mid L)=0.85,\\
    P(S)=0.40, & P(s_1 \mid S)=0.79, & P(s_2 \mid S)=0.21.
    \end{matrix}
    \]
    What is the optimal decision strategy if the family purchases the fast-growing seeds so they can wait and learn more about the rabbit appetites before making a decision?
    
    \item If \$40\ of the fast-growing seed purchase is non-refundable, should the family purchase the fast-growing seeds? Why or why not? What is the maximum non-refundable amount the family should pay to get the fast-growing seeds?
\end{enumerate}

\begin{solution}
\bs For part a), the expected savings when planting the garden are
\[ 400 \times 0.3  - 100 \times 0.7 = \$50. \]
The savings from not planting the garden are \$0, so based on expected value,
the best decision is to plant the garden.

For part b), if the rabbits have large appetites ($L$), then planting 
the garden would result in -\$25 of expected savings.
If the rabbits have small appetites ($S$), then planting the garden will result in \$295 
of expected savings. 

If L, \[ \$100 \times 0.15 - \$100 \times 0.85 = -\$25 \]
If S, \[ \$400 \times 0.79 - \$100 \times 0.21 = \$295 \]

Not planting will always result in \$0 of savings. 
The optimal decision strategy is to plant the garden if $S$ and buy vegetables 
from the store if $L$.

For part c), we use the optimal decision for each possible event $L$ and $S$. 
The expected savings from purchasing the fast-growing seeds
(but before actually purchasing the seeds) are
\[ \$0 \times 0.60 + \$295 \times 0.40 = \$118 \]
The maximum non-refundable amount that the family should be willing 
to pay for the fast-growing seeds is
\[ \$118 - \$50 = \$68 \]
\end{solution}

% this problem is OK
\item \emph{Using Baye's formula to update a prior belief.}
Curling is a sport in which players slide a stone over ice toward a
target. The association governing the sport has implemented drug
testing. It is believed that 15\% of all curlers use banned drugs to
enhance performance. If a player uses banned drugs, the association
may take away any prizes that the player has won; however, it
undesirable to falsely accuse someone of using banned substances.  The
utilities for each decision and state of nature are

\begin{center}
\begin{tabular}{rrrr}
& drug use & no drug use \\ \cline{2-3}
take away prizes & -100 & -1000 \\
do not & -600 & 0 
\end{tabular}
\end{center}

Notice that there is a small dis-utility for taking prizes away from a drug user
due to bad publicity for the sport. The test to detect drug use is
less than 100\% reliable. In particular, if $D$ indicates that a 
player uses banned drugs, and $+$/$-$ indicate a positive/negative
test result, then the true positive rate and the true negative rate
are
\[
P(+ \mid D) = .97 \quad \text{and} \quad P(- \mid \overline{D}) = .97,
\]
respectively, Given the utilities and the accuracy of the test, what is the best
decision if a player has a positive test result? (The association
wants to maximize expected utility.)

\begin{solution}
\bs First we update the probability of drug use via Baye's formula.
\begin{align*}
P(D \mid +) &= \frac{P(D \cap +)}{P(+)} \\
&= \frac{P(+ \mid D)P(D)}{P(+ \mid D)P(D) + P(+ \mid \overline{D})P(\overline{D})} \\
&= \frac{.97 \times .15}{.97 \times .15 + .03 \times .85} \\
&= .851
\end{align*}
and then we can compute $P(\overline{D} \mid +) = 1 - P(D \mid +) = .149$.
Using these posterior probabilities, the expected utilities are
\begin{align*}
E(\text{take away}) &= (-100)(.851) + (-1000)(.149) = -234 \\
E(\text{do not}) &= (-600)(.851) = -511
\end{align*}
The best decision is to take away prizes when a player tests positive.
\end{solution}

% written by Emily
\item \emph{Elimination of dominated strategies}
Two street vendors, A and B, are located near a major tourist attraction. 
The proportion of customers
captured by each vendor depends on the merchandise sold by that vendor and by
her competitor. A customer gained by one is lost to the other. Each vendor
can stock one of the following: clothing, ice cream, or souvenirs.
The possible strategies and proportion of customers captured are as follows.

\begin{tabular}{l}
If both shops sell souvenirs, A captures 75\% of the customers.\\
If both shops sell clothing, A and B split the customers evenly.\\
If both shops sell ice cream, A and B split the customers evenly.\\
If B sells ice cream and A sells souvenirs, A captures 10\%.\\
If B sells clothing and A sells ice cream, A captures 90\%.\\
If B sells souvenirs and A sells clothing, A captures 10\%.\\
If A sells clothing and B sells ice cream, A captures 100\%.\\
If A sells souvenirs and B sells clothing, A captures 75\%.\\
If A sells ice cream and B sells souvenirs, A captures 40\%.
\end{tabular}

\setlength{\parindent}{0cm}
Model the decision of each vendor as two-person zero-sum game
and find a solution by elimination of dominated strategies.

\begin{solution}
\bs The game is

\begingroup
\setlength{\tabcolsep}{9pt}
\renewcommand*{\arraystretch}{2}
\begin{tabularx}{4.5in}{YYYYY}
& & \multicolumn{3}{c}{B} \\
& & clothing & ice cream & souvenirs \\ \cline{3-5}
\multirow{3}{.25in}{A} & \gtcol{clothing} & \gtcol{.50} & \gtcol{1} & \gtcol{.10} \\ \cline{3-5}
& \gtcol{ice cream} & \gtcol{.90} & \gtcol{.50} & \gtcol{.40} \\ \cline{3-5}
& \gtcol{souvenirs} & \gtcol{.75} & \gtcol{.10} & \gtcol{.25} \\ \cline{3-5}
\end{tabularx}
\endgroup
\vspace{.1in}

For A, ice cream strictly dominates souvenirs and for B, souvenirs strictly dominates
clothing, leaving a $2 \times 2$ game.

\begingroup
\setlength{\tabcolsep}{9pt}
\renewcommand*{\arraystretch}{2}
\begin{tabularx}{4in}{YYYY}
& & \multicolumn{2}{c}{B} \\
& & ice cream & souvenirs \\ \cline{3-4}
\multirow{2}{.25in}{A} & \gtcol{clothing} & \gtcol{1} & \gtcol{.10} \\ \cline{3-4}
& \gtcol{ice cream} & \gtcol{.50} & \gtcol{.40} \\ \cline{3-4}
\end{tabularx}
\endgroup
\vspace{.1in}

Now for B, souvenirs dominates ice cream. Then, A will choose to sell ice cream
over clothing. So, the
best strategies for A and B are to sell ice cream and souvenirs, respectively. 
Using this pair of strategies, A will capture 40\% of the customers.
\end{solution}

\end{enumerate}
