\chapter{Data Analysis}
\label{data-analysis}


\section{Descriptive Statistics}

Point estimate of a population proportion. Consider the
  pizza delivery data that is available on the class Moodle
  page. Construct a point estimate of the probability that the amount
  of tips received in a shift is greater than \$60. What is the
  standard error of your point estimate? You can do the calculations
  by hand or use the software of your choice. If you use software, you
  can use it in any way you like.  For example, I used R as a
  calculator to simply help with the required computations.
  
\begin{Verbatim}
pizza <- read.table("pizza.txt", header=TRUE)
attach(pizza)
x <- sum(Tips > 60)
n <- length(Tips)

# follow the formula for the point estimate and the standard
# error of a sample proportion
phat <- x/n
se <- sqrt((phat*(1-phat))/n)

> phat
[1] 0.2413793
> se
[1] 0.0212373
\end{Verbatim}


\section{Descriptive Graphics}

\section{Exercises}

\begin{enumerate}
\subsubsection*{Descriptive Statistics}

%re-written
\item \emph{Point estimate of a population mean.} Suppose the
  following data points are a sample of a golfer's scores over his last 20 rounds.
  Construct a point estimate of his average score. What is the standard error of your point estimate? You can do
  this problem either by hand or use the software of your choice. 
\begin{verbatim}
73,69,65,70,67,67,78,72,74,71,70,69,70,67,68,73,70,77,72,69
\end{verbatim}
  
%re-written
\item {Standard error when estimating a proportion.} In a survey, a
  random sample of \num{1200} students are asked whether they prefer online or in-person classes. 
  Out of the \num{1200} students, \num{424} said they prefer online classes. Compute a point estimate 
  of the overall proportion of students that prefer online classes and calculate the standard error of your estimate. 

\subsubsection*{Descriptive Graphics}

% this problem is OK.
\item \emph{College students and driving speed.} The file
  \texttt{speed\_gender\_height.csv} contains 1,325 observations on
  gender, height, and the fastest speed ever driven (in mph) for a
  sample of college students.

\begin{enumerate}
\item Create a boxplot of speed by gender. That is to say, make one
  boxplot for males and one boxplot for females, but put them
  side-by-side on the same plot.
\item Make an x--y plot with height on the x--axis and speed on the
  y--axis. Color the plotted points according to gender. Place a
  legend that shows the color associations. Another option is to use
  different plotting symbols rather than color to distinguish males
  and females.
\end{enumerate}

\end{enumerate}
