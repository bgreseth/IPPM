\chapter{Data Analysis}
\label{data-analysis}

\section{Descriptive Statistics}

Point estimate of a population proportion. Consider the
  pizza delivery data that is available on the class Moodle
  page. Construct a point estimate of the probability that the amount
  of tips received in a shift is greater than \$60. What is the
  standard error of your point estimate? You can do the calculations
  by hand or use the software of your choice. If you use software, you
  can use it in any way you like.  For example, I used R as a
  calculator to simply help with the required computations.
  
\begin{Verbatim}
pizza <- read.table("pizza.txt", header=TRUE)
attach(pizza)
x <- sum(Tips > 60)
n <- length(Tips)

# follow the formula for the point estimate and the standard
# error of a sample proportion
phat <- x/n
se <- sqrt((phat*(1-phat))/n)

> phat
[1] 0.2413793
> se
[1] 0.0212373
\end{Verbatim}


\section{Descriptive Graphics}

\section{Exercises}

\begin{enumerate}
\subsubsection*{Descriptive Statistics}

%re-written
\item \emph{Point estimate of a population mean.} Suppose the
  following data points are a sample of a golfer's scores over his
  last 20 rounds.  Construct a point estimate of his average
  score. What is the standard error of your point estimate? You can do
  this problem either by hand or use the software of your choice.
\begin{verbatim}
73,69,65,70,67,67,78,72,74,71,70,69,70,67,68,73,70,77,72,69
\end{verbatim}
  
%re-written
\item \emph{Standard error when estimating a proportion.} In a survey, a
  random sample of \num{1200} students are asked whether they prefer
  online or in-person classes.  Out of the \num{1200} students,
  \num{424} said they prefer online classes. Compute a point estimate
  of the overall proportion of students that prefer online classes and
  calculate the standard error of your estimate.



% this problem is OK
\item \emph{Lognormal distribution parameter estimation.}  The file
  \texttt{component-lifetimes.txt} contains the time to failure for
  \num{1345} components (in hours). The times are known to come from a
  Lognormal distribution. Estimate the parameters of the failure time
  distribution. Use the parameters to estimate the mean time to
  failure and also to estimate the probability that a component lasts
  longer than 10,000 hours.

% this problem is OK
\item \emph{Required sample size.} The following data are observations
  from a past study of hummingbird migration rates in miles flown per
  day. These observations are from 30 different birds. A researcher
  would like to construct a two-sided, 95\% confidence interval for
  the average rate (in miles per day). The researcher would like for
  the width of the confidence interval to be no larger than 1 day. It
  is very expensive to attach identifiers to the birds, and so the
  researcher has asked you to determine the smallest sample size that
  will achieve the desired confidence interval. What sample size do
  you suggest?
\begin{Verbatim}[samepage=true]
17,17,22,18,19,21,21,23,21,25,
19,21,19,20,20,21,19,20,18,17,
18,20,19,23,18,22,18,22,18,24
\end{Verbatim}

\begin{solution}
\bs The width of a confidence interval is
2 times the half-width
\[
L = 2 \times z_{\alpha/2} \times s/\sqrt{n}
\]
and the researcher would like $L\leq 1$. We can use the data to
estimate a standard deviation $\hat{s} = 2.133$. We do not know $n$. In
fact, that is the question we are trying to answer. You can assume
that you will have a sample size at least as big as, say, 30
birds. Using $n=30$ and the tabulated values for the Normal
distribution, we find that $z_{\alpha/2} = z_{.025} = 1.96$.  Then the
required sample size is
\[
n \geq 4 \times (z_{\alpha/2} \times \hat{s})^2 = 
4 \times (1.96 \times 2.133)^2 = 69.9
\]
We would recommend a sample size no smaller than 70 observations.
\end{solution}

\subsubsection*{Descriptive Graphics}

% this problem is OK.
\item \emph{College students and driving speed.} The file
  \texttt{speed\_gender\_height.csv} contains 1,325 observations on
  gender, height, and the fastest speed ever driven (in mph) for a
  sample of college students.

\begin{enumerate}
\item Create a boxplot of speed by gender. That is to say, make one
  boxplot for males and one boxplot for females, but put them
  side-by-side on the same plot.
\item Make an x--y plot with height on the x--axis and speed on the
  y--axis. Color the plotted points according to gender. Place a
  legend that shows the color associations. Another option is to use
  different plotting symbols rather than color to distinguish males
  and females.
\end{enumerate}

\end{enumerate}
