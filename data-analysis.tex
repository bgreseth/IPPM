\chapter{Data Analysis}
\label{data-analysis}

\section{Descriptive Statistics}

Point estimate of a population proportion. Consider the
  pizza delivery data that is available on the class Moodle
  page. Construct a point estimate of the probability that the amount
  of tips received in a shift is greater than \$60. What is the
  standard error of your point estimate? You can do the calculations
  by hand or use the software of your choice. If you use software, you
  can use it in any way you like.  For example, I used R as a
  calculator to simply help with the required computations.
  
\begin{Verbatim}
pizza <- read.table("pizza.txt", header=TRUE)
attach(pizza)
x <- sum(Tips > 60)
n <- length(Tips)

# follow the formula for the point estimate and the standard
# error of a sample proportion
phat <- x/n
se <- sqrt((phat*(1-phat))/n)

> phat
[1] 0.2413793
> se
[1] 0.0212373
\end{Verbatim}

\emph{Confidence Interval using the $t$ Distribution.}
An article in the \emph{Journal of Heat Transfer} describes a
method of measuring the thermal conductivity of high-purity iron.
Using a temperature of $100^{\circ}$F and a power input of
550 W. The following 10 measurements of thermal conductivity
(in Btu/hr -- ft -- ${}^{\circ}$F) were determined.
\begin{align*}
  41.60,~41.48,~&42.34,~41.95,~41.86\\
  42.18,~41.72,~&42.26,~41.81,~42.04
\end{align*}
A point estimate of the population mean thermal conductivity
(at $100^{\circ}$F and 550 W) is the sample mean
\[
  \mean{X} = 41.924
\]
The standard error of the sample mean (i.e. the standard error
of the point estimate) is
\[
  \text{se}(\mean{X}) = \frac{S}{\sqrt{n}} = \frac{0.284}{\sqrt{10}} = 0.0898
\]
where $S$ is the sample standard deviation. Notice that the standard
error is about 0.2 percent of the sample mean, indicating a relatively
precise point estimate of thermal conductivity. Occasionally you will
hear people refer to the coefficient of variation (CV).
\[
  \text{CV} = \frac{S}{\mean{X}}
\]
The CV is another measure of the spread of the data. \emph{Question:}
What are the units of the CV?

Now suppose we want to construct a 95\% confidence interval for the
population mean thermal conductivity $\mu$. So, our confidence level
will be $1-\alpha=.95$, and $\alpha=.05$. Since there are only 10
sample data points, we will use the $t$ distribution. From our
discussion in class we know that a $(1-\alpha)$\% confidence interval
for $\mu$ is
\[ \mean{X} \pm t_{\alpha/2,n-1}\frac{S}{\sqrt{n}} \]
From the tabulated values of the $t$ distribution, we see that
$t_{\alpha/2,n-1} = t_{.025,9} = 2.262$. A 95\% confidence interval for $\mu$ is
\[  41.924 \pm 2.263 \times \frac{0.284}{\sqrt{10}} \]
or (41.721, 42.127).

Using R, we could do the following

\vspace{.1in}
\begin{Verbatim}[samepage=true]
> x <- c(41.60,41.48,42.34,41.95,41.86,42.18,41.72,42.26,41.81,42.04)
> alpha <- .05
> n <- length(x)
> xbar <- mean(x)
> se <- sd(x)/sqrt(n)
> cp <- qt(1-alpha/2,n-1)

> xbar + c(-1,1)*cp*se
[1] 41.72076 42.12724
\end{Verbatim}
\vspace{.1in}
To use the Normal distribution instead of the $t$ distribution, obtain the
critical point as
\begin{Verbatim}
> cp <- qnorm(1-alpha/2)
\end{Verbatim}

\emph{Question:} When I got the critical point in R, why did I use
\texttt{qnorm(1-alpha/2)} and not \texttt{qnorm(alpha/2)}?

\section{Descriptive Graphics}

\section{Exercises}

\begin{enumerate}
\subsubsection*{Descriptive Statistics}

% this problem needs a different data set. The questions are fine, but
% we cannot use this data set.
\item \emph{Summarizing a data set with statistics.} Each time a
  particular supermarket receives a shipment of peaches, the manager
  chooses one box at random and counts the number of spoiled peaches
  in the box. (Each box contains 48 peaches.) The file
  \texttt{fruit-spoilage.txt} contains data on the number of spoiled
  peaches for 55 such boxes.

\begin{enumerate}
\item From the data set of 55 observations, determine the following
summary statistics: minimum, maximum, mean, median, mode, first quartile,
third quartile, and standard deviation. To be clear, from this data set
you are computing sample statistics; the true underlying distribution
of the number of spoiled peaches is unknown.
\item Create a table the shows the distinct values of the number of 
spoiled peaches and their counts, i.e. the number of times that
each value occurred.
\end{enumerate}

% re-written
\item \emph{Point estimate of a population mean.} Suppose the
  following data points are a sample of a golfer's scores over his
  last 20 rounds. Construct a point estimate of his average
  score. What is the standard error of your point estimate?
\begin{verbatim}
73,69,65,70,67,67,78,72,74,71,70,69,70,67,68,73,70,77,72,69
\end{verbatim}
  
% re-written
\item \emph{Standard error when estimating a proportion.} In a survey, a
  random sample of \num{1200} students are asked whether they prefer
  online or in-person classes.  Out of the \num{1200} students,
  \num{424} said they prefer online classes. Compute a point estimate
  of the overall proportion of students that prefer online classes and
  calculate the standard error of your estimate.

% this problem is OK
\item \emph{Lognormal distribution parameter estimation.}  The file
  \texttt{component-lifetimes.txt} contains the time to failure for
  \num{1345} components (in hours). The times are known to come from a
  Lognormal distribution. 
\begin{enumerate}
\item Estimate the parameters of the failure time distribution. 
\item Use the parameters to estimate the mean time to failure. 
\item Use the parameters to estimate the probability that a component lasts
  longer than \num{10000} hours.
\end{enumerate}

% this problem is OK
\item \emph{Confidence interval for a mean.} Restaurants are making
  more use of their data on service times for planning purposes.  The
  data in the file \texttt{restaurant-service-times.txt} contains 220
  observations on the time in minutes from seating until departure in
  one particular restaurant. 
\begin{enumerate}
\item Construct a 95\% confidence interval for the true mean service
  time. Remember that this data is just a sample from a larger
  population, the true distribution of which is unknown.
\item Do you think that the Central Limit Theorem applies to this
  data? Why or why not?
\end{enumerate}

% this problem is OK
\item \emph{Required sample size.} The following data are observations
  from a past study of hummingbird migration rates in miles flown per
  day. These observations are from 30 different birds. A researcher
  would like to construct a two-sided, 95\% confidence interval for
  the average rate (in miles per day). The researcher would like for
  the width of the confidence interval to be no larger than 1 day. It
  is very expensive to attach identifiers to the birds, and so the
  researcher has asked you to determine the smallest sample size that
  will achieve the desired confidence interval. What sample size do
  you suggest?
\begin{Verbatim}[samepage=true]
17,17,22,18,19,21,21,23,21,25,
19,21,19,20,20,21,19,20,18,17,
18,20,19,23,18,22,18,22,18,24
\end{Verbatim}

\begin{solution}
\bs The width of a confidence interval is
2 times the half-width
\[
L = 2 \times z_{\alpha/2} \times s/\sqrt{n}
\]
and the researcher would like $L\leq 1$. We can use the data to
estimate a standard deviation $\hat{s} = 2.133$. We do not know $n$. In
fact, that is the question we are trying to answer. You can assume
that you will have a sample size at least as big as, say, 30
birds. Using $n=30$ and the tabulated values for the Normal
distribution, we find that $z_{\alpha/2} = z_{.025} = 1.96$.  Then the
required sample size is
\[
n \geq 4 \times (z_{\alpha/2} \times \hat{s})^2 = 
4 \times (1.96 \times 2.133)^2 = 69.9
\]
We would recommend a sample size no smaller than 70 observations.
\end{solution}

\subsubsection*{Descriptive Graphics}

% this problem is OK.
\item \emph{College students and driving speed.} The file
  \texttt{speed\_gender\_height.csv} contains 1,325 observations on
  gender, height, and the fastest speed ever driven (in mph) for a
  sample of college students.

\begin{enumerate}
\item Create a boxplot of speed by gender. That is to say, make one
  boxplot for males and one boxplot for females, but put them
  side-by-side on the same plot.
\item Make an x--y plot with height on the x--axis and speed on the
  y--axis. Color the plotted points according to gender. Place a
  legend that shows the color associations. Another option is to use
  different plotting symbols rather than color to distinguish males
  and females.
\end{enumerate}

\end{enumerate}
