\chapter{Data Analysis}
\label{data-analysis}


\section{Descriptive Statistics}

Point estimate of a population proportion. Consider the
  pizza delivery data that is available on the class Moodle
  page. Construct a point estimate of the probability that the amount
  of tips received in a shift is greater than \$60. What is the
  standard error of your point estimate? You can do the calculations
  by hand or use the software of your choice. If you use software, you
  can use it in any way you like.  For example, I used R as a
  calculator to simply help with the required computations.
  
\begin{Verbatim}
pizza <- read.table("pizza.txt", header=TRUE)
attach(pizza)
x <- sum(Tips > 60)
n <- length(Tips)

# follow the formula for the point estimate and the standard
# error of a sample proportion
phat <- x/n
se <- sqrt((phat*(1-phat))/n)

> phat
[1] 0.2413793
> se
[1] 0.0212373
\end{Verbatim}


\section{Descriptive Graphics}

\section{Exercises}

\begin{enumerate}
\subsubsection*{Descriptive Statistics}

% this problem needs to be re-written
\item \emph{Point estimate of a population mean.} Suppose the
  following 20 data points are a sample of service times (in minutes)
  from a restaurant. Construct a point estimate of the average service
  time. What is the standard error of your point estimate? You can do
  this problem either by hand or use the software of your choice. If
  you use software, you can use it in any way you like. For example, I
  used R as a calculator to simply help with the required
  computations.
\begin{verbatim}
59,45,62,52,72,91,88,64,65,69,59,70,63,80,70,59,87,59,69,68
\end{verbatim}
  
% this problem needs to be re-written
\item \emph{Maximum error when estimating a proportion.} In a poll a
  random sample of \num{1400} residents are asked whether they support
  or oppose a proposal. What is the largest possible value of the
  standard error of the estimate of the overall proportion in favor of
  the proposal?

\begin{solution}
  \bs Let $\hat{p}$ be the estimate of the true overall proportion of
  people in favor of the proposal. An unbiased point estimate for $p$
  is
\[ \hat{p} = \frac{X}{1400} \]
where $X$ is the number of people who responded to the poll that they
favor the proposal. $X$ follows a Binomial distribution. Now,
\begin{align*}
  \text{Var}(\hat{p}) = \frac{1}{n^2}\text{Var}(X) &= \frac{1}{n^2}np(1-p) \\
                                                   &= \frac{p(1-p)}{n} \\
                                                   &= \frac{p(1-p)}{1400}
\end{align*}
The variance (and hence the standard error) is maximum when the
quantity $p(1-p)$ is maximum, or when $p=1/2$. Using $p=1/2$ and
taking the square root we get a maximum standard error of
\[ \sqrt{\frac{.5(1-.5)}{1400}} \approx 0.013 \]

\end{solution}

\subsubsection*{Descriptive Graphics}

% this problem is OK.
\item \emph{College students and driving speed.} The file
  \texttt{speed\_gender\_height.csv} contains 1,325 observations on
  gender, height, and the fastest speed ever driven (in mph) for a
  sample of college students.

\begin{enumerate}
\item Create a boxplot of speed by gender. That is to say, make one
  boxplot for males and one boxplot for females, but put them
  side-by-side on the same plot.
\item Make an x--y plot with height on the x--axis and speed on the
  y--axis. Color the plotted points according to gender. Place a
  legend that shows the color associations. Another option is to use
  different plotting symbols rather than color to distinguish males
  and females.
\end{enumerate}

\end{enumerate}
